\RequirePackage{etex} 
\documentclass[xcolor=svgnames]{beamer}
\usetheme{Singapore}
\setbeamertemplate{navigation symbols}{}
%\setbeamertemplate{footline}{framenumber}
%\setbeamertemplate{page number in head/foot}[framenumber]
%\setbeamertemplate{page number in head/foot}[noframenumbering]
%\setbeamertemplate{footline}[frame number]
\setbeamercolor{block body}{bg=AliceBlue}

\usepackage{accents}
\usepackage{amsmath,amssymb}
\usepackage{array}
\usepackage{bibentry}
\usepackage{bm}
%\usepackage{cancel}
\usepackage{mathscinet}
\usepackage{mathtools}
\usepackage{rotating}
%\usepackage{tikz}

\hypersetup{
    colorlinks=true,
    linkcolor=blue,
    filecolor=magenta,      
    urlcolor=cyan
    }

    %% Gianni macros

\DeclareMathOperator{\Hajek}{Hajek}
\DeclareMathOperator{\Maxexp}{\mathcal E}
\DeclareMathOperator{\M}{\mathcal M}
\DeclareMathOperator{\N}{\mathcal N}
\DeclareMathOperator{\eDeriv}{D_{\text{e}}}
\DeclareMathOperator{\Grad}{grad}
\DeclareMathOperator{\mDeriv}{D_{\text{m}}}

\newcommand{\Bspaceat}[1]{L^2_0(#1)}
\newcommand{\Derivby}[1]{\frac{\Deriv}{d#1}}
\newcommand{\KL}[2]{\operatorname{D}\left(#1\,\Vert#2\right)}
\newcommand{\Lexp}[1]{L^{(\cosh-1)}\left(#1\right)}
\newcommand{\LlogL}[1]{L^{(\cosh-1)_*}\left(#1\right)}
\newcommand{\TMaxexp}{\operatorname{T}\Maxexp}
\newcommand{\WCexp}[1]{C_0^{1,(\cosh-1)}\left( #1 \right)}
\newcommand{\Wexp}[1]{W^{1,(\cosh-1)}\left(#1\right)}
\newcommand{\WlogL}[1]{W^{1,(\cosh-1)_*}\left(#1\right)}
\newcommand{\avalof}[1]{\left\vert#1\right\vert}
\newcommand{\bgamma}{{\bm \gamma}}
\newcommand{\bnabla}{{\bm\nabla}}
\newcommand{\condexpat}[3]{\mathbb E_{#1}\left(#2 \middle| #3\right)}
\newcommand{\derivby}[1]{\frac{d}{d#1}}
\newcommand{\displacement}{\operatorname{\mathbb S}}
\newcommand{\eBspace}[1]{B_{#1}}
\newcommand{\eDerivby}[1]{\frac{\eDeriv}{d#1}}
\newcommand{\ehessianat}[2]{\prescript{e}{}\Hessian_{#1}{#2}}
\newcommand{\etransport}[2]{\prescript{\text{e}}{} {\mathbb U} _ {#1} ^ {#2}}
\newcommand{\euler}{\mathrm{e}}
\newcommand{\expbundleat}[1]{S#1}
\newcommand{\expectat}[2]{\mathbb E_{#1}\left[#2\right]}
\newcommand{\expfibreat}[2]{S_{#1}#2}
\newcommand{\expof}[1]{\exp\left(#1\right)}
\newcommand{\fullbundleat}[1]{\prescript{1}{}S^1\maxexpat{#1}}
\newcommand{\gaussdensity}{\gamma}
\newcommand{\gaussint}[2]{\int{#1} \gaussdensity(#2) \ d#2 \ }
\newcommand{\hajekof}[1]{\Hajek\left(#1\right)}
\newcommand{\hullof}[1]{\operatorname{hull}\left(#1\right)}
\newcommand{\logof}[1]{\log\left(#1\right)}
\newcommand{\mDerivby}[1]{\frac{\mDeriv}{d#1}}
\newcommand{\maxexpat}[1]{\Maxexp\left(#1\right)}
\newcommand{\maxmix}[1]{\prescript{*}{}{\Maxexp\left(#1\right)}}
\newcommand{\mhessianat}[2]{\prescript{m}{}\Hessian_{#1}{#2}}
\newcommand{\mixbundleat}[1]{\prescript{*}{}S\maxexpat{#1}}
\newcommand{\mixbundle}{\prescript{*}{}S\Maxexp}
\newcommand{\mixfiberat}[2]{{}^*S_{#1}\maxexpat{#2}}
\newcommand{\model}{\mathcal M}
\newcommand{\mtransport}[2]{\prescript{\text{m}}{} {\mathbb U} _ {#1} ^ {#2}}
\newcommand{\openplan}[2]{\overset{\circ}\Pi\left(#1,#2\right)}
\newcommand{\opensimplexat}[1]{\mathcal P_>\left(#1\right)}
\newcommand{\opensimplexon}[1]{\mathcal P_>\left(#1\right)}
\newcommand{\reals}{\mathbb{R}}
\newcommand{\rosso}[1]{\textcolor{red}{#1}}
\newcommand{\scalarat}[3]{\left\langle#2,#3\right\rangle_{#1}}
\newcommand{\sdomainat}[1]{\sdomain_{#1}}
\newcommand{\sdomain}{\mathcal S}
\newcommand{\setof}[2]{\left\{#1 \, \middle| \, #2 \right\}}
\newcommand{\set}[1]{\left\{#1\right\}}
\newcommand{\simplexon}[1]{\mathcal P\left(#1\right)}
\newcommand{\tensorat}[3]{\prescript{#1}{}S^{#2}\maxexpat{#3}}
\newcommand{\transport}[2]{{\mathbb U} _ {#1} ^ {#2}}
\newcommand{\velocity}[1]{\accentset{\star}{#1}}
\renewcommand{\emph}{\rosso}

%
%% end Gianni's macros
%

\title{B0262 \\ \it Information geometry of ANOVA and transport on a finite state space}

\author[G Pistone]{Giovanni Pistone}

\institute[CCA]{\href{https://www.carloalberto.org/research/statistics-initiative/}{\includegraphics[height=4em]{pictures/deCastro-logo.pdf}}}
  
\date{Compiled \today}

\begin{document} 
% 
\begin{frame}\frametitle{EO357 \href{http://www.cmstatistics.org/CMStatistics2023/}{CMStatistics 2023}}  

\titlepage

\tiny 
web-page: \url{www.giannidiorestino.it} 

e-mail: \url{giovanni.pistone@carloalberto.org}

orcidID: \href{https://orcid.org/0000-0003-2841-788X}{0000-0003-2841-788X}

The author acknowledges the support of \href{https://www.carloalberto.org/research/statistics-initiative/}{De Castro Statistics}. He is a member of \href{https://www.carloalberto.org/research/statistics-initiative/}{INdAM-GNAMPA} and \href{https://aiml.unich.it}{UMI-AI\&ML\&MAT}.

\nobibliography{tutto}%

\bibliographystyle{plain}

\end{frame}

\begin{frame}[plain]\small\frametitle{Abstract}

  \emph{Functional ANOVA} (Analysis of variance) appears in Statistics and System Theory. It is a particular orthogonal splitting of the vector space of square-integrable random variables on a product space. When the sample space is factorial,  it conveniently splits the fibres of the affine bundle consisting of couples of probability functions and Fisher's scores, which we call the \emph{statistical bundle}. One of the terms in the splitting is the \emph{additive model}, while the other is related to the \emph{transportation model} with fixed margins. This concept is known in the classical theory of contingency tables. We rephrase it and show implications to algebraic statistics, information geometry, and Kantorovich optimal transport. In this setting, the \emph{gradient flow} in the transport sub-model has a limit point that solves the \emph{Kantorovich problem}.

  \vfill
\emph{keywords}: ANOVA, statistical bundle, gradient flow, additive and transportation model, Kantorovich problem. 

\end{frame}

\begin{frame}[plain]\small\frametitle{References}
\tiny 

\begin{description}
    \item[1968] \bibentry{hajek:1968}
    \item[1981] \bibentry{efron|stein:1981variance}
    \item[2001] \bibentry{sobol:2001global}
    \item[2021] \bibentry{pistone:2021gsi}
    \item[2021] \bibentry{pistone|rapallo|rogantin:2021}
    \item[2022] \bibentry{chirco|pistone:2022}
\end{description}
\end{frame}

\begin{frame}[plain]

\huge PART 1 \\ ANOVA and AFFINE STATISTICAL BUNDLE.
    
\end{frame}

    \begin{frame}[plain,allowframebreaks]\small\frametitle{ANOVA with two non-independent factors}

    \begin{itemize}
    \item Consider a finite product sample space $\Omega = \Omega_1 \times \Omega_2$. The generic probability function is denoted
    \begin{equation*}
       q \colon  \Omega_1 \times \Omega _2 \ni (x_1,x_2) \mapsto q(x_1,x_2) \ . 
    \end{equation*}
    We denote the two margins by 
    \begin{equation*}
    q_1(x_1) = \sum_y f(x_1,y) \ , \quad q_2(x_2) = \sum_x q(x,x_2) \ .
    \end{equation*}
    \item Normally, we do not use a numbering $\Omega_k = \set{1,\dots,n_k}$ and do not identify the probability function $q$ with a matrix (or table) $Q \in \reals^{n_1 \times n_2}$.
    \item For each probability function $q$ and each random variable $f \in L^2(q)$ we look for $q$-orthogonal decomposition of the form
    \begin{equation*}
        f(x_1,x_2) = f_0 \oplus (f_1(x) + f_2(x_2)) \oplus f_{12}(x_1,x_2)
    \end{equation*}
    \item Notice that we do not require $f_1 \perp f_2$ as it is done in case of independence, cf Hajek:1968 and Sobol':2001.
    \item We call \emph{factors} the two marginal projections
\begin{equation*}
    X_1 \colon (x_1,x_2) \mapsto x_1 \ , \quad X_2 \colon (x_1,x_2) \mapsto x_2 \ .  
    \end{equation*}
  \item Consider the subsets $I \subset \set{1,2}$, partially ordered by inclusion, that is, 
\begin{equation*}\emptyset \prec \set{1},\set{2} \prec \set{1,2} \ .
\end{equation*}
\item  Each $I \neq \emptyset$ is an \emph{interaction}. 
   Let $X_I$ be the components projection on $I$, $X_I = (X_j \colon j \in I)$.
    \item A $q$-effect is a random variable with zero $Q$-mean. A \emph{$q$-effect of the interaction $I$} is a $q$-effect of the form $f\circ X_I$ which is $q$-orthogonal to all $g \circ X_J$ for all $J \prec I$, that is, $J \subset I$ and $J \neq I$.
    \item The \emph{order} of the interaction $I$ is $\# I$. Let $H_k$ be the vector space generated by the $I$-interactions of order $k$. $H_0$ contains random variables which do not depend on any $X_j$ $j=1,2$. that is, $H_0 = \reals$.
    \item The space $H_1$ is generated by the random variables of the form 
        $f_1\circ X_1$ and $f_2\circ X_2$
    with 
    \begin{gather*}
        \expectat q {f_1\circ X_1} = \expectat {q_1} {f_1} = 0 \\
           \expectat q {f_2\circ X_2} = \expectat {q_2} {f_2} = 0 
    \end{gather*}
   \item An element of $H_1$ is of the form
   \begin{equation*}
       f_1\circ X_1 + f_2 \circ X_2 \ , \quad f_1 \in L^2_0(q_1) \ , f_2 \in L^2_0(q_2)
   \end{equation*}
   \item The representation above is unique. In fact, if
   \begin{equation*}
       f_1(x_1) + f_2(x_2) = 0 \ , x_1 \in \Omega_1, x_2 \in \Omega_2 \ ,
   \end{equation*}
   then both $f_1$ and $f_2$ must be constant. As the $q$-expectation is zero, $f_1 = f_2 = 0$.
   \item An element of $H_2$ is of the form $f_{12} \circ(X_1,X_2)$ with
   \begin{equation*}
     f_{12} \circ(X_1,X_2) \perp H_{\emptyset}, H_{\set{1}}, H_{\set{2}}  
   \end{equation*}
   The orthogonality with respect to $H_{\emptyset}$ implies zero $q$-expectation $\expectat q {f_{12}} = 0$. The orthogonality with respect to $H_{\emptyset} + H_{\set{1}}$ and $H_{\emptyset} + H_{\set{2}}$ implies zero conditional expectation with respect to each factor:
   \begin{equation*}
       \condexpat {q} {f_{12} \circ (X_1,X_2)}{X_1} = 0 \ , \quad \condexpat {q} {f_{12} \circ (X_1,X_2)}{X_2} = 0
   \end{equation*}
    \item We have a $q$-orthogonal decomposition of $f \in L^2(q)$ of the form
    \begin{equation*}
    0 = f_0 \oplus (f_1\circ X_1 + f_2\circ X_2) \oplus f_{12}\circ (X_1,X_2)
    \end{equation*}
    with $f_0 \in H_0$, $(f_1\circ X_1 + f_2\circ X_2) \in H_1$, and $f_{12}\circ (X_1,X_2) \in H_2$. 
    \item Let $f \mapsto \hajekof q f$ be the orthogonal projection of $L^2(q)$ onto $H_1$, the \emph{Hajek projection}.
    \item The orthogonal decomposition of $f \in L^2(q)$ is computed as
    \begin{equation*}
        f = \expectat q f \oplus \hajekof q f \oplus (\operatorname I  - 
 \mathbb E_q - \hajekof q) f \ .
    \end{equation*}
    \item We want to compute the Hajek projection. It is a least square problem
      \begin{equation*}
        \begin{array}{ll}
    \text{minimize} & \expectat q {\avalof {f - f_0 - f_1\circ X_1 - f_2\circ X_2}^2} \\    
    \text{with}& f_0 \in \reals, \expectat {q_1} {f_1} = 0, \expectat {q_2} {f_2} = 0.
    \end{array}
      \end{equation*}
    \item The following equations are the normal equations of the least square problem. Or, they follow from the conditioning of the decomposition.
    \begin{align*}
        \expectat q {f} &= f_0 \\
        \condexpat q {f}{X_1} &= f_0 + f_1\circ X_1 + \condexpat q {f_2 \circ X_2}{X_1} \\ 
            \condexpat q {f}{X_2} &= f_0 + \condexpat q {f_1 \circ X_1}{X_2} + f_2\circ X_2
        \end{align*}
\end{itemize}
\end{frame}

\begin{frame}[plain,allowframebreaks]\small\frametitle{Affine statistical bundle}
\begin{itemize}
    \item The \emph{affine statistical bundle} is a structure that describes the joint geometry of probabilities and random variables. This justifies the adjective "statistical".
\item The geometry is affine in the sense of Weyl's axioms: for each couple of points $P, Q \in \model$ there is vector $v = \overrightarrow{PQ}$ in such a way $Q = P+v$ and $\overrightarrow{PQ}+\overrightarrow{QR}=\overrightarrow{PR}$. 

\item We consider the set of couples $(q,v)$ such that $q$ is a positive probability function, $q \in \opensimplexon \Omega$, and $v$ is a random variable whose $q$-expectation is zero, $v \in L^2_0(q)$. The vector space $L_0^2(q)$ is the \emph{fibre} at $q$.

\item We modify the original Weyl's definition to allow for vector spaces depending on the base point.
%\item The affine charts define a special atlas whose change-of-charts mappings are affine.
\end{itemize} 

\begin{definition}[Statistical bundle]
    \begin{equation*}
        S\opensimplexon \Omega = \setof{(q,v)}{q \in \opensimplexon \Omega, v \in L^2_0(q)}
    \end{equation*}
\end{definition}
\end{frame}

\begin{frame}[plain,allowframebreaks]\small\frametitle{Exponential chart}

  \begin{itemize}

   \item We define the \emph{exponential displacement} from $p \in \opensimplexon \Omega$ to $q \in \opensimplexon \Omega$ by
    \begin{equation*}
(p,q) \mapsto s_p(q) = \log \frac q  p - \expectat p {\log \frac q p} \in L^2_0(p) = \expfibreat p {\opensimplexon \Omega} \ ,
    \end{equation*}
    and the \emph{exponential transport} between fibres  by
    \begin{equation*}
      \etransport q p \colon \expfibreat q {\opensimplexon \Omega} \ni v \mapsto v - \expectat p v \in \expfibreat p {\opensimplexon \Omega} \ .
    \end{equation*}
\item The parallelogram law holds true:
    \begin{multline*}
      \left(\log \frac q p - \expectat p {\log \frac q p}\right) + \etransport q p \left(\log \frac r q - \expectat q {\log \frac r q}\right) = \\   \left(\log \frac q p - \expectat p {\log \frac q p}\right) + \left(\log \frac r q - \expectat p {\log \frac r q}\right) = \\ \log \frac r p - \expectat p {\log \frac r p} \ . 
    \end{multline*}
\item The inverse chart (the patch) $s_p^{-1}$ is defined on all of the fibre $\expfibreat p {\opensimplexon \Omega}$ by
  \begin{equation*}
    s^{-1}_p(v) = \expof{v - K_p(v)} \cdot p = e_p(v) \ , \quad K_p(v) = \log \expectat p {\euler^v} \ .
  \end{equation*}
\item The \emph{cumulant functional}
\begin{equation*}
   K_p \colon \expfibreat p {\opensimplexon \Omega} \ni v \mapsto K_p(v) = \log \expectat p {\euler^v}
 \end{equation*}
 has several important properties.
 \item It is an expression in the affine chart of the \emph{Kullback-Leibler divergence} as a function of the second variable. If $s_p(q) = v$, then
\begin{equation*}
    \KL p q = \expectat p {\log \frac p q} = \expectat p {\log \frac p {\expof{v - K_p(v)} \cdot p}} = K_p(v) \ .
\end{equation*}
\end{itemize}
\end{frame}

\begin{frame}[plain]\small\frametitle{Mixture chart}
\begin{itemize}
\item  We define the \emph{mixture displacement} from $p \in \opensimplexon \Omega$ to $q \in \opensimplexon \Omega$ on by
       \begin{equation*}
   (p,q) \mapsto \eta_p(q) = \frac q p - 1 \in \Bspaceat p = \expfibreat p {\opensimplexon \Omega} \ ,
    \end{equation*}
    and the \emph{mixture transport} between fibres by
    \begin{equation*}
      \mtransport p q \colon \expfibreat p {\opensimplexon \Omega} \ni v \mapsto \frac p q \, v \in \expfibreat p {\opensimplexon \Omega} \ .
    \end{equation*}
\item The (generalized) parallelogram law holds true
  \begin{equation*}
 \left(\frac q p -1\right) + \frac q p \left(\frac r q -1\right) = \left(\frac r p -1\right) \ .
\end{equation*}
\item The inverse chart $\eta_p(v)$ is defined for all $v > -1$, $w \in \expfibreat p {\opensimplexon \Omega}$ by
  \begin{equation*}
    \eta^{-1}_p(v) =  (1+v) \cdot p \ .
  \end{equation*}
\end{itemize} 

\end{frame}

\begin{frame}[plain,allowframebreaks]\small
  \frametitle{Duality, velocity, gradient}
  \begin{itemize}
  \item The exponential transport and the mixture transport are \emph{dual} of each other with respect to the $L^2_0$ inner product, $\scalarat p v w = \expectat p {v w}$. For all $v \in \expfibreat q {\opensimplexon \Omega}$ and $w \in \expfibreat p {\opensimplexon \Omega}$ it holds
  \begin{equation*}
    \scalarat q v {\etransport p q w} = \scalarat p {\mtransport q p v} w \ . 
  \end{equation*}
\item The \emph{velocity  in the chart at $p$} of a smooth curve $t \mapsto q(t) \in \opensimplexon \Omega$ is
  \begin{gather*}
    \derivby t s_p(q(t)) = \derivby t \left(\log \frac  {q(t)} p  - \expectat p {\frac  {q(t)} p }\right) = \frac{\dot q(t)} {q(t)} - \expectat p {\frac{\dot q(t)} {q(t)}} \ ,\\ \text{or} \quad
    \derivby t \eta_p(q(t)) = \derivby t \left(\frac {q(t)} p -1\right) = \frac {\dot q(t)} p \ .
    \end{gather*}
  \item In the \emph{moving frame} $p = q(t)$ the two representation are equal. Such an expression of the velocity,
    \begin{equation*}
      \velocity q(t) = \frac {\dot q(t)}{q(t)} = \derivby t \log q(t) \ ,  
    \end{equation*}
    equals the classical \emph{Fisher's score}. Notice that $\velocity q$ is a lift to the bundle, $t \mapsto (q(t), \velocity q(t)) \in \expbundleat {\opensimplexon \Omega}$.
  \item The \emph{(natural) gradient} of a smooth function $\phi \colon \opensimplexon \Omega \to \reals$ is the section $\Grad \phi$ of the statistical bundle such that for all smooth curve $t \mapsto q(t)$ it holds
    \begin{equation*}
      \derivby t \phi(q(t)) = \scalarat {q(t)} {\Grad \phi(q(t))} {\velocity q(y)} \ .
    \end{equation*}
    \item The \emph{gradient flow} of $\phi$ is the solution of the equation
      \begin{equation*}
        \velocity q(t) = - \Grad \Phi(q(t)) \ .
      \end{equation*}
  \end{itemize}
\end{frame}

\begin{frame}[plain]

    \huge PART 2: \\
    PRODUCT SAMPLE SPACE
    
\end{frame}



\begin{frame}[plain,allowframebreaks]\small\frametitle{Fixed margins}
\begin{itemize}
    \item Assume a product sample space $\Omega = \Omega_1 \times \Omega_2$ and consider the probability simplex $\simplexon{\Omega_1 \times \Omega_2}$. The two \emph{marginalisation} mappings are
    \begin{align*}
 \Pi_1& \colon \simplexon{\Omega_1 \times \Omega_2} \ni q \mapsto \sum_{x_2 \in \Omega_2} q(\cdot,x_2) \in  \simplexon{\Omega_1}  \\
 \Pi_2& \colon \simplexon{\Omega_1 \times \Omega_2} \ni q \mapsto \sum_{x_1 \in \Omega_1} q(x_1,\cdot) \in  \simplexon{\Omega_2} 
    \end{align*}
    \item Each $q$ is a \emph{transport plan} from $q_1$ to $q_2$.
    \item For each given $q_1 \in \simplexon{\Omega_1}$ and $q_2 \in \simplexon{\Omega_2}$ define 
    \begin{equation*}
        \Pi(q_1,q_2) = \setof{q \in \simplexon{\Omega_1 \times \Omega_2}} {\Pi_1 q = q_1, \Pi_2 q = q_2} \ .
    \end{equation*}
    \item The set of transport plans  $\Pi(q_1,q_2)$ is non-empty, convex, and closed.
      \item The marginalization conditions
    \begin{gather*}
        \sum_{x_2 \in X_2} q(y_1,x_2) = q_1(y_1) \quad y_1 \in \Omega_1 \\
            \sum_{x_1 \in X_1} q(x_1,y_2) = q_2(y_2) \quad y_2 \in \Omega_2
    \end{gather*}
    can be written as
    \begin{gather*}
        \sum_{(x_1,x_2) \in X_1\times X_2} (y_1 = x_1) q(x_1,x_2) = q_1(y_1) \\
             \sum_{(x_1,x_2) \in X_1\times X_2} (y_2 = x_2) q(x_1,x_2) = q_2(y_2)
    \end{gather*}
    thus identifying an operator $A \colon (\Omega_1 \cup \Omega_2) \times (\Omega_1 \times \Omega_2)$ with
    \begin{equation*}
     \sum_{(x_1,x_2) \in \Omega_1 \times \Omega_2} A(y;x_1,x_2) q(x_1,x_2) = (q_1 \cup q_2)(y) \ , \quad y \in \Omega_1 \cup \Omega_2 
    \end{equation*}
\end{itemize}    
\end{frame}

\begin{frame}[plain]\small\frametitle{Transport plans in $\opensimplexon{\Omega_1 \times \Omega_2}$}
 \begin{itemize}
 
     \item For all $q_1 \in \opensimplexon{\Omega_1}$ and $q_2 \in \opensimplexon{\Omega_2}$ the set of \emph{positive transport plans} from $q_1$ to $q_2$ is
     \begin{equation*}
         \openplan{q_1}{q_2} = \setof{q \in \opensimplexon{\Omega_1 \times \Omega_2}}{\Pi_1 q = q_1,\Pi_2 q= q_2}
     \end{equation*}
 
   \item A \emph{sub-manifold} of the affine statistical manifold $(\M,s_p,\eBspace p,\transport p q \colon p,q \in M)$ is a subset $\N \subset \M$ such that for each $q \in \N$ there exists a smooth \emph{splitting} of the fibre at $q$,
     \begin{equation*}
       \eBspace q = S_q \N \oplus R_q \N \ ,
     \end{equation*}
     and the vector space $S_q \N$ is the set of all velocities of curves in $\N$ through $q$. 

\item Basic examples of sub-manifolds of the affine statistical manifold are exponential families and mixture models. Notice that a sub-manifold of the affine statistical manifold is not forced to be an affine space.

\item $\openplan{q_1}{q_2}$ is a \emph{sub-manifold} of the affine statistical manifold on $\opensimplexon{\Omega_1 \times \Omega_2}$.

\end{itemize}
\end{frame}

\begin{frame}[plain,allowframebreaks]\small\frametitle{Velocity of a curve in $\openplan{q_1}{q_2}$}

\begin{itemize}

\item Let $t \mapsto q(t)$ be a smooth curve of $\opensimplexon {\Omega_1 \times \Omega_2}$ with values in the set of strictly positive transport plans, 
      $      t \mapsto q(t) \in \openplan{q_1}{q_2}$.
      
    \item Recall \emph{Fisher's score} properties,
    \begin{gather*}
        \velocity q(t) = \derivby t \log q(t) = \frac {\dot q(t)}{q(t)} \\
        \derivby t \expectat {q(t)} f = \scalarat {q(t)} {f - \expectat {q(t)} f}{\velocity q(t)} \ . 
    \end{gather*}
    \item For each random variable depending only on one factor
      \begin{multline*} 0 = \derivby t \expectat {q_j} {f_j} =  \derivby t \expectat {q(t)} {f_j\circ X_j} = \\ \scalarat {q(t)}  {f_j \circ X_j - \expectat {q(t)} {f_j\circ X_j}} {\velocity q(t)} = \expectat {q(t)} {f_j\circ X_j \velocity q(t)} \ . \end{multline*}
Hence $\condexpat {q(t)} {\velocity q(t)} {X_j} = 0$, $j=1,2$.
\item  That is, \emph{$\velocity q(t)$ is a  $q(t)$-interaction}, $\velocity q(t) \in H_2(q(t))$.
\item Conversely, let $q \in \openplan{q_1}{q_2}$ and $c_{12} \in H_2(q)$. The curve $t \mapsto (1+tc_{12}) \cdot q$ is defined for $t$ in a neighborhood of 0, stays in $\openplan{q_1}{q_2}$,
\begin{equation*}
    \expectat {(1+tc_{12}) \cdot q}{g\circ X_j} = \expectat q {(1+tc_{12})g\circ X_j} = \expectat {q_j} g \ ,
\end{equation*}
and the velocity at 0 is $c_{12}$,
\begin{equation*}
   \left. \derivby t \logof{(1+tc_{12}) \cdot q}
\right|_{t=0} = \left. \frac {c_{12} q}{(1+tc_{12})q} \right|_{t=0} = c_{12} \ \end{equation*}
\end{itemize}
\begin{block}{Proposition}For all $q \in \openplan{q_1}{q_2}$, the velocities' fibre equals the vector space of interactions,
\begin{equation*}
S_q \openplan{q_1}{q_2} = H_2(q)
\end{equation*}
\end{block}

\begin{itemize}\item A splitting of the statistical bundle at $q \in \openplan{q_1}{q_2}$ is
\begin{equation*}
    \expfibreat q \opensimplexon{\Omega_1 \times \Omega_2} = S_q \openplan{q_1}{q_2} \oplus \hajekof{q}\expfibreat q \opensimplexon{\Omega_1 \times \Omega_2} \ .
\end{equation*}

\item The complement fibre $R_q \expfibreat q {\opensimplexon{\Omega_1 \times \Omega_2}}$ is 
\begin{multline*}
 H_1(q) =  \hajekof{q}\expfibreat q \opensimplexon{\Omega_1 \times \Omega_2} = \\ \setof {f_1 \circ X_1 + f_2 \circ X_2}{\expectat {q_1}{f_1} = \expectat {q_2}{X_2} = 0} \ ,
 \end{multline*}
  which in turn provides the \emph{exponential family of additive statistics}, 
  \begin{equation*}
      \expof{f_1 \circ X_1 + f_2 \circ X_2 - K_q(f_1 \circ X_1 + f_2 \circ X_2)} \cdot q \ .
  \end{equation*}
\end{itemize}
\end{frame}

\begin{frame}[plain,allowframebreaks]\small\frametitle{$\openplan{q_1}{q_2}$ as an affine space}
\begin{itemize}
\item If $q , r \in \openplan {q_1} {q_2}$ and $c_{12} \in H_2(q) = S_q \openplan {q_1}{q_2}$,
\begin{equation*}
\expectat r {\mtransport q r c_{12} g_i \circ X_i} = \expectat r {\left(\frac q r c_{12}\right) g_i \circ X_i} = \expectat q {c_{12} \,  g_i \circ X_i} = 0
\end{equation*}
that is, $\frac q r c \in H_2(r) = S_r \openplan {q_1}{q_2}$.
\item We have defined a co-cycle 
of parallel transports on the bundle
\begin{equation*}
    S \openplan {q_1}{q_2} = \setof {(q,c)}{ q \in \openplan {q_1}{q_2}, c \in H_2(q)} 
\end{equation*}
\item The dual transport is computed as follows. If $q, r \in \openplan {q_1}{q_2}$ and
\begin{equation*}
  c_{12} \in S_q \openplan {q_1}{q_2} = H_2(q) \ , \quad 
  d_{12} \in S_r \openplan {q_1}{q_2} = H_2(r) \ ,
\end{equation*}
then
\begin{equation*}
\scalarat r {\mtransport q r c} d = \expectat q {cd} = \scalarat q c {d - \hajekof q d}
\end{equation*}
\item Let us compute the mixture geodesic. If $(q,c) \in S \openplan {q_1}{q_2}$, an m-geodesic is a curve in $t \mapsto q(t) \in \openplan {q_1}{q_2}$ such that $(q(0),\velocity q(0)) = (q,c)$ and $\velocity q(t) = \mtransport q {q(t)} c$. It follows
\begin{equation*}
    \frac {\dot q(t)}{q(t)} = \frac q {q(t)} c \quad \text{then} \quad q(t) = (1 +tc) \cdot q \ .
\end{equation*}
\emph{The m-geodesic from $q$ in the direction $c$ is $t \mapsto (1+tc)\cdot q$}. 
\item The \emph{the affine displacement is the geodesic at $t=1$}:
\begin{equation*}
 \openplan {q_1}{q_2} \times  \openplan {q_1}{q_2} \ni (q,r) \mapsto \frac r q - 1
\end{equation*}
\item The e-geodesic from $q$ in the direction $c$ is the solution of
\begin{equation*}
\velocity q(t) = (I - \hajekof{q(t)})c \ .
\end{equation*}
\item A solution of this equation seems to require a solution of the Hajek projection.
\end{itemize}

\end{frame}

\begin{frame}[plain]\small\frametitle{Gradient of the expected cost}

Let us discuss the \emph{Optimal Transport OT} problem in the framework of the affine statistical bundle.

\begin{itemize}
\item $c \colon \Omega_1 \times \Omega_2 \to \reals_{\ge}$ is the \emph{cost function} and the expected cost function is
\begin{equation*}
  C \colon \opensimplexon {\Omega_1 \times \Omega_2} \ni q \mapsto \expectat q c \ .
\end{equation*}
\item The function $q \mapsto C(q)$ restricted to the open transport model $q \in \openplan{q_1}{q_2}$ has \emph{gradient in $S\openplan{q_1}{q_2}$} given by
\begin{multline*}
\derivby t C(q(t)) = \derivby t \expectat {q(t)} c = \scalarat {q(t)} {c - \expectat {q(t)} c}{\velocity q(t)} = \\
\scalarat {q(t)} {\left(c - \expectat {q(t)} c\right) - \hajekof {q(t)} \left(c - \expectat {q(t)} c\right)}{\velocity q(t)} \ ,
\end{multline*}
that is,
\begin{equation*}
  \Grad C(q) = \left(c - C(q(t))\right) - \hajekof {q(t)} \left(c - C(q(t))\right)
\end{equation*}
\end{itemize}
\end{frame}

\begin{frame}[plain]\small
  \frametitle{Gradient flow of the OT cost}
  \begin{itemize}
  \item The equation of the \emph{gradient flow of $C$} is
\begin{equation*}
  \velocity q(t) = - \left(c - C(q(t)) - \hajekof {q(t)} \left(c - C(q(t))\right)\right) \ .
\end{equation*}

\item Notice that the gradient above is the projection onto the space orthogonal to the space of simple effects. Hence, it is actually well defined for all $q \in \simplexon{\Omega_1 \times \Omega_2}$. If $\hat q$ is a zero of this extended map, then $c$ equals the sum of two functions in one variable on the support of $\hat q$.
\item
  \emph{If} a solution $t \mapsto q(t)$ of the gradient flow equation converges to a transport plan $\bar q = \lim_{t \to \infty} q(t) \in \Pi(q_1,q_2)$, then $\expectat {\bar q} c$ is the value of the Kantorovich optimal transport problem.
\item The form of the gradient is compatible with the classical result in OT: if $\bar q$ is an optimal plan, that the cost is equal to the sum of two univariate potentials. 
\end{itemize}
\end{frame}

\end{document} % Appendix follows

\begin{frame}[plain]

    \huge APPENDIX
    
  \end{frame}

  \begin{frame}[plain,allowframebreaks]\small\frametitle{Kantorovich optimal transport}
\begin{itemize}
    \item Given the \emph{cost} $c \colon \Omega_1 \times \Omega_2 \to \reals$, Kantorovich looks for the transport plan with minimal expected cost
    \begin{equation*}
        \inf \setof{\sum_{x_1,x_2} c(x_1,x_2) q(x_1,x_2)}{q \in \Pi(q_1,q_2)}
    \end{equation*}
    \item It is a \emph{primal problem} in canonical form:
       \begin{align*}
      \text{Find} \quad c =& \inf \sum_{x_1,x_2} c(x_1,x_2) q(x_1,x_2) \\
      \text{Subject to} \quad &\sum_x A(y;x_1,x_2) q(x_1,x_2) = (q_1 \cup q_2)(y) \\
      &q(x_1,x_2) \geq 0 \ .
    \end{align*}

    \item The \emph{dual problem} in standard form is
\begin{align*}
      \text{Find} \quad \beta =& \sup \left(\sum_{y_1\in X_1}q_1(y_1) \lambda_1(y_1) + \sum_{y_2 \in X_2} q_2(y_2) \lambda_2(y_2)\right) \\
      \text{Subject to} \quad &\lambda_1(x_1) + \lambda_2(x_2) \leq c(x_1,x_2) \ .    \end{align*}

    \item There exists a feasible transport plan, and the set of plans is compact. It follows that the full strong duality theorem holds. Let $\bar q$, $\bar \lambda$ be the optimal plan and dual plan. The equality of values gives
    \begin{equation*}
    \sum_{x_1,x_2} c(x_1,x_2)\bar q(x_1,x_2) = \sum_{x_1,x_2} \left(\bar \lambda_1(x_1) + \bar \lambda_2(x_2)\right) \bar q(x_1,x_2) 
    \end{equation*}
    \item Finally, the inequality $c \geq \lambda_1 \oplus \lambda _2$ implies
    \begin{equation*}
        c(x_1,x_2)=\bar \lambda_1(x_1) + \bar \lambda_2(x_2) \quad \text{provided $\bar q(x_1,x_2) \neq 0$}
    \end{equation*}
\end{itemize}
    
\end{frame}

\end{document}

\begin{frame}[plain,allowframebreaks]\small\frametitle{Example} 
 
 %\begin{figure}
  \begin{tabular}{ccc}
    \begin{tikzpicture}[scale=1.7,baseline = (current bounding box.center)]
\node (n00) at (1,-.5) {$\delta_{11}$};
\node (n10) at (0,0) {$\delta_{21}$};
\node (n01) at (2,0) {$\delta_{12}$};
\node (n11) at (1,1.5) {$\delta_{22}$};
\node (K1) at (5/6,-1/12) {$\gamma_1$};
\node (K2) at (5/6,1/4) {$\gamma_2$};
\draw[thick,dotted] (n01) -- (n10);
\foreach \from/\to in {n00/n10,n00/n01,n11/n10,n11/n01,n00/n11}
\draw[thick] (\from) -- (\to);
\draw [thick,red] (K1) -- (K2); % was: gray, dashed
\end{tikzpicture} & $\xrightarrow[(X_1,X_2)_{\#}]{}$ &
    \begin{tikzpicture}[scale=3,baseline = (current bounding box.center)]
\node (n00) at (0,0) {$\delta_{11}$};
\node (n10) at (1,0) {$\delta_{12}$};
\node (n01) at (0,1) {$\delta_{21}$};
\node (n11) at (1,1) {$\delta_{22}$};
\foreach \from/\to in {n00/n01,n01/n11,n11/n10,n10/n00}
\draw[thick] (\from) -- (\to);
\filldraw [red] (1/2,2/3) circle (1pt); % was: gray
\end{tikzpicture}
  \end{tabular}
  %\caption{
  %\label{fig:2x2}}
  %\end{figure}

Cf Pistone-Rapallo-Rogantin:2021
\begin{itemize}
\item The sample space is: 
\begin{equation*}
  \Omega_1=\Omega_2 = \set{1,2} \quad \Omega_1 \times \Omega_2 = \set{11,12,21,22} 
\end{equation*}
\item The margins are:
\begin{equation*}
   {\color{red}\bullet} = \left((1/2, 1/2), (2/3, 1/3)\right) = \left((q_1(1),q_1(2)),(q_2(1),q_2(2))\right) 
\end{equation*}

\item Cf. Exercise 2x2 coupling.

\item The vertexes of the transport polytope are:
\begin{equation*}
\gamma_1 = \begin{bmatrix} 1/6 & 1/3 \\ 1/2 & 0 \end{bmatrix} \quad 
\gamma_2 = \begin{bmatrix} 1/2 & 0 \\ 1/6 & 1/3 \end{bmatrix}\end{equation*}
\item The transport polytope is 
\begin{align*}
 \openplan{q_1}{q_2} &= \setof{(1-t)\begin{bmatrix} 1/6 & 1/3 \\ 1/2 & 0 \end{bmatrix} + t \begin{bmatrix} 1/2 & 0 \\ 1/6 & 1/3 \end{bmatrix}}{0 < t < 1} \\
 &= \setof{\begin{bmatrix} 1/6 & 1/3 \\ 1/2 & 0 \end{bmatrix} + t \begin{bmatrix} 1/3 & -1/3 \\ -1/3 & 1/3 \end{bmatrix}}{0 < t < 1}
\end{align*}
\item A curve in the transport polytope is 
\begin{equation*}
   \gamma \colon t \mapsto \begin{bmatrix} 1/6 & 1/3 \\ 1/2 & 0 \end{bmatrix} + a(t) \begin{bmatrix} 1/3 & -1/3 \\ -1/3 & 1/3 \end{bmatrix}
\end{equation*}

\end{itemize}
\end{frame}

 

\begin{frame}[plain,allowframebreaks]\small\frametitle{Aside on Linear Programming LP}

\begin{itemize}  \item See \S IV.8 in \bibentry{barvinok:2002}
\end{itemize}

  \begin{itemize}
  \item The \emph{primal problem in canonical form} is
    \begin{align*}
      \text{Find} \quad c =& \inf \sum_x c(x) r(x) \\
      \text{Subject to} \quad &\sum_x A(y,x) r(x) = \beta(y) \quad y \in Y \\
      &r(x) \geq 0
    \end{align*}
  \item $r$ is the \emph{primal plan}
  \item a plan is \emph{feasible} if the constraints hold
  \end{itemize}

  \begin{itemize}
  \item The \emph{dual problem in standard form} is
    \begin{align*}
      \text{Find} \quad \beta =& \sup \sum_y \beta(y) \lambda(y) \\
      \text{Subject to} \quad &\sum_y A(y,x) \lambda(y) \leq c(x) 
    \end{align*}
  \item $\lambda$ is the \emph{dual plan}
  \end{itemize}

  \begin{block}{Strong duality theorem}
  If a feasible primal plan exists, then $c = \beta$. If moreover, $c > -\infty$, then primal optimal and dual optimal plans exist.  
  \end{block}
\end{frame}

\begin{frame}[plain]\frametitle{Tangent bundle of the positive orthant of the sphere}

\includegraphics[width=\textwidth]{exercise/positive-orthant-cropped.pdf}

\end{frame}

\begin{frame}[plain]\frametitle{Tangent bundle as an atlas}

\includegraphics[width=\textwidth]{exercise/atlas-cropped.pdf}

\end{frame}

\begin{frame}[plain]\frametitle{Square-root-likelihood}

\includegraphics[width=\textwidth]{exercise/square-root-cropped.pdf}

\end{frame}

\begin{frame}[plain]\frametitle{Fisher-Cramer-Rao trick}

\includegraphics[width=\textwidth]{exercise/fisher-cramer-rao.pdf}

\end{frame}

\begin{frame}[plain]

  \includegraphics[height=.8\textheight]{exercise/scheme.pdf}
  
\end{frame}

\begin{frame}[plain]\frametitle{Exercise: conditional expectation}

\includegraphics[width=\textwidth]{exercise/bayes-conditional-expectation.pdf}

\end{frame}

\begin{frame}[plain,allowframebreaks]\frametitle{Exercise: $2\times2$ coupling}

\includegraphics[width=\textwidth]{exercise/2x2-coupling-1.jpg}

\includegraphics[width=\textwidth]{exercise/2x2-coupling-2.jpg}
    
\end{frame}

\end{document}

%%% Local Variables:
%%% reftex-default-bibliography: ("/home/giannidiorestino/Dropbox/InProgress/tutto.bib")
%%% TeX-master: t
%%% End:

https://www.overleaf.com/project/64109af0a2c0904d6c5e12e8
